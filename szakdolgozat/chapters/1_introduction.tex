\Chapter{Bevezetés}


Manapság a böngészőprogramokat elég sokat használjuk, híreket olvasunk, információkat gyűjtünk, videókat nézünk, egyre több hivatalos ügyet is online intézünk, és játékokkal is játszunk \cite{alk}. Ezekhez tipikusan valamilyen webes alkalmazást használunk. Ezekből kifolyólag a böngészők a webes technológiákkal együtt modern alkalmazásfejlesztési platformmá váltak.

Először is kezdjük ott, hogy mit is nevezünk böngészőnek vagy webböngészőnek? Jellemzően azokat a programokat nevezzük, amelyekkel az Interneten található úgynevezett weblapokat lehet megtekinteni.
Vannak ismertebb böngészőprogramok is, melyek a következőek:
\begin{itemize}
	\item Google Chrome,
	\item Mozzila Firefox,
	\item Opera,
	\item Safari,
	\item Microsoft Internet Explorer,
	\item Microsoft Edge.
\end{itemize}
Ezek közül néhány már kezd kikopni a használatból, ilyen például a Microsoft Internet Explorer.

Programozási szempontból a böngészők, már nagyon sokféle szolgáltatást tudnak nyújtani. Kényelmesen és eseményvezérelt módon tudjuk kezelni benne a sokféle grafikus elemet, amit képesek megjeleníteni. Ezen alkalmazások nagy előnye, hogy bárhonnan (böngésző programot támogató mobil vagy számítógép), bármikor (általában internetkapcsolattal) elérhetőek. 

Nekünk ezek közül a szolgáltatások közül a játékok lesznek érdekesek, pontosabban azok grafikus megjelenítése. Ugyanis a dolgozat témája a folyadékok modellezése, ami a játékokban lehet leginkább fontos.

Manapság az alkalmazások egyre jobban megkövetelik a közel valós idejű információ elérést \cite{valos}. Ilyenek például a közösségi oldalak, de a több szereplős játékok is, és ha már az információk elérése valós idejű, akkor a játékok megjelenítésének is valós idejűen kell lennie.

A dolgozatban a víz megjelenítéséhez egy modellezési forma fog bemutatásra kerülni, annak kétféle változata. Ez a részecske rendszer alapú (tulajdonképpen véges elem) módszerekre épülnek.
A modellben meg kell határozni, hogy hány darab részecskére van szükség, hogy megfelelően ábrázolható legyen a program. A kirajzolás ne legyen lassú, és a program könnyedén tudjon futni. Ehhez tesztek is készültek, mert 2 ábrázolási forma is szóba jött. Erről bővebben a 5. fejezetben lesz szó. Vizsgálni kell az ütközéseket, meghatározni az ütközési pontotokat, majd azután meghatározni a vektorokat, hogy merre folytatja útját a részecske és milyen sebességgel. De ezeknek a pontosabb leírására a dolgozat további részeiben kerül majd sor részletesen. 
