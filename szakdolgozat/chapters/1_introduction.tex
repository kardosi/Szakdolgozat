\Chapter{Bevezetés}

A dolgozat témája a folyadékok modellezése. Két modellezési forma fog bemutatásra kerülni. Az egyik modell részecske alapú, a másik cella alapú. Mindkét modellnek vannak érdekességei, nehézségei. A részecske alapúban meg kell határozni, hogy hány darab részecskére van szükség, hogy megfelelően ábrázolható legyen. A kirajzolás ne legyen lassú, és a program könnyedén tudjon futni. Vizsgálni kell az ütközéseket, majd azután, hogy merre mozdulna a részecske az ütközés után. Majd a feladatot optimalizálni is szükséges, mert sok részecskénél nagyon lassú lenne a program, ha az összes részecskére néznénk az ütközéseket. A cella alapúban pedig azt kell vizsgálni, hogy egy adott cella üres-e vagy sem. Kitöltünk cellákat, majd nézzük, hogy melyik üres, és ha üres és felette tele van, akkor az alsó lesz tele és a felső üres. 