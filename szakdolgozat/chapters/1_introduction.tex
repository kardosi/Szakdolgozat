\Chapter{Bevezetés}

A fejezet célja, hogy a feladatkiírásnál kicsit részletesebben bemutassa, hogy miről fog szólni a dolgozat.
Érdemes azt részletezni benne, hogy milyen aktuális, érdekes és nehéz probléma megoldására vállalkozik a dolgozat.

Ez egy egy-két oldalas leírás.
Nem kellenek bele külön szakaszok (section-ök).
Az irodalmi háttérbe, a probléma részleteibe csak a következő fejezetben kell belemenni.
Itt az olvasó kedvét kell meghozni a dolgozat többi részéhez.

\textbf{Leírás}

A folyadékok modellezése egy rég óta kutatott terület. A fizikai jelenség és a technológiai limitációk egyaránt komplikálttá teszik a problémát. A dolgozat célja annak bemutatása, hogy hogyan lehet valós időben szimulálni a különféle folyadékok mozgását és milyen formában lehet azt egy böngészőben megjeleníteni. Ehhez meg kell vizsgálni a korábbi, hasonló célú alkalmazásokat. Meg kell adni a folyadék mozgásának leírásához a megfelelő matematikai modellt. Részletezni kell, hogy milyen feltételezések mellett, milyen optimalizációs módszerekkel oldható meg a valós idejű megjelenítés. A számítások elvégzéséhez JavaScript programozási nyelv, a megjelenítéshez pedig HTML Canvas elem kerül majd felhasználásra.
