\Chapter{Bevezetés}

Manapság a böngészőprogramokat elég sokat használjuk, híreket olvasunk, információkat gyűjtünk, videókat nézünk, egyre több hivatalos ügyet is online intézünk, és játékokkal is játszunk.Ezekhez valamilyen webes alkalmazást használunk. Ezekből kifolyólag a böngészők a webes technológiákkal együtt modern alkalmazásfejlesztési platformmá váltak. 

Programozási szempontból a böngészők, már nagyon sokféle szolgáltatást tudnak nyújtani. Kényelmesen és eseményvezérelt módon tudjuk kezelni benne a sokféle grafikus elemet, amit képesek megjeleníteni. Ezen alkalmazások nagy előnye, hogy bárhonnan (böngésző programot támogató mobil vagy számítógép), bármikor (általában internetkapcsolattal) elérhetőek. 

Nekünk ezek közül a szolgáltatások közül a játékok lesznek érdekesek, pontosabban a grafikus megjelenítés. Ugyanis a dolgozat témája a folyadékok modellezése, ami a játékokban lehet fontos. 

Manapság az alkalmazások egyre jobban megkövetelik a közel valós idejű információ elérést. Ilyenek például a közösségi oldalak, de a több szereplős játékok is. És ha már az információk elérése valós idejű, akkor a játékok megjelenítésének is valós idejűen kell lennie.

A dolgozatban a víz megjelenítéséhez két modellezési forma fog bemutatásra kerülni. Az egyik modell részecske alapú, a másik cella alapú. Mindkét modellnek vannak érdekességei, nehézségei. 


A részecske alapúban meg kell határozni, hogy hány darab részecskére van szükség, hogy megfelelően ábrázolható legyen. A kirajzolás ne legyen lassú, és a program könnyedén tudjon futni. Ehhez tesztek is készültek, mert 2 ábrázolási forma is szóba jött, erről bővebben a 5. fejezetben lesz szó. Vizsgálni kell az ütközéseket, meghatározni az ütközési pontot, majd azután meghatározni a vektorokat, hogy merre folytatná útját a részecske és milyen sebességgel.

A cella alapúban pedig azt kell vizsgálni, hogy egy adott cella üres-e vagy sem. Kitöltünk cellákat, majd nézzük, hogy melyik üres, és ha üres és fölötte tele van, akkor az alsó lesz tele és a felső üres. 

