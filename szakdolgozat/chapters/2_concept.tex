\Chapter{Koncepció}

\Section{Elterjedt szimulációs módszerek}


\textbf{Végeselem}

A véges elem módszerekkel
bonyolult feladatokat egyszerűsítünk le, ezt hálózásnak nevezzük. Ennek lényege, hogy az elemeket véges számú kisebb, egyszerűbb elemekre bontjuk, így több, de egyszerűbb számításokat kell elvégeznünk, a kevesebb, de bonyolultabb számítások helyett.
A módszer elterjedésével megváltoztak a gyártási folyamatok. Ezeket a módszereket jellemzően mérnökök használják, mechanikai számítások elvégzéséhez, valamint mezőszimulációkhoz. Tipikus példa egy bonyolult alakú statikusan terhelt gépalkatrész feszültségi állapotának és alakváltozásának meghatározása. Ekkor a az alkatrész modellt véges számú elemre bontják. Ilyenek a síkbeliek közül a háromszögek és a négyszögek, térbeliek közül pedig a hasábok és tetraéderek. A felosztást úgy érdemes csinálni, hogy ahol az eredmény kritikus lehet ott kisebb részekre osztani, máshol pedig nagyobb részekre. Az elemek a csomópontoknál csatlakoznak egymáshoz. A véges elem módszert még nemlineáris feladatoknál is alkalmazni lehet. Például nemlineáris anyagtulajdonságok, nagy deformációk kezelése, szerkezeti stabilitási problémák. Hóvezetési problémák és kombinált hővezetési és szilárdsági problémák megoldására is használják a módszert. A mágneses és elektrosztatikus problémákra is megfelelő a használata. Mostanában szivárgási és talajmechanikai vizsgálatokat is végeznek vele. Azonban a véges elem módszer kézi számításokra nem alkalmas, mert nagyon sok elemi műveletet kellene végezni. A kis teljesítményű személyi számítógépek azonban alkalmasak, hogy megoldható legyen egy sor gyakorlatilag fontos feladat. Nem csak a matematikai feladat megoldása munkaigényes, hanem maga az adatok előkészítése, és az eredmények értékelhető alakba hozása is. Ezért egy preprocesszor és egy posztprocesszor is a korszerű számítógépes programok részét képezik.   




\Section{Részecske alapú megközelítés}


A részecske alapú megközelítés lényege, hogy a vizet, mint kis részecskék vagyis körök ábrázoljuk. A program működésének alapelve, hogy random ledobunk kis részecskéket és nézzük, hogy hol ütköznek, majd megvizsgáljuk, hogy az ütközés után merre kell haladniuk. 

Minden részecskének van színe, mérete, ami az összesre nézve azonos. 

A megvalósításhoz 3 osztályt használunk, ami a pont, részecske és maga a szimuláció. A pont osztályban a pontok tulajdonságai vannak. A részecske osztályban  a részecskék tulajdonságai és kirajzolása. A szimulátor osztályban pedig maga a szimulátor, a részecskék létrehozásával, a kép frissítéssel és az ütközésvizsgálattal. 

\Section{Cella alapú módszer}

Rácsfelbontás, most praktikusan egy négyzetrács

Mátrixban ábrázolható

Minden cellának van egy kitöltöttség értéke, például $[0, 1]$, ahol 0 az üres, 1 a teljesen kitöltött.

Jelölni kell a falakat is, például negatív vagy valamilyen nagyobb értékkel.
