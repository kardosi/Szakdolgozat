\Chapter{Koncepció}

\Section{Elterjedt szimulációs módszerek}

Végeselem módszerek (Finite Element Methods)

\textbf{Végeselem}

A végeselemes módszer (VEM) numerikus módszer parciális differenciálegyenletek közelítő megoldására. Jellemzően mérnökök használják, a gépészmérnöki és építőmérnöki feladatokkal szorosan összefüggő mechanikai (szilárdságtani és lengéstani) számítások elvégzésére, valamint a villamosmérnöki gyakorlatban mezőszimuláció során is ezt a módszert alkalmazzák.

A végeselemes módszert alkalmazni lehet nemlineáris feladatok megoldására is, mint például nemlineáris anyagtulajdonságok, nagy deformációk kezelésére, szerkezeti stabilitási problémák megoldására (épületszerkezetek stabilitása, kihajlása, külső nyomásnak kitett tartályok horpadása). Használják a VEM programokat hővezetési problémák illetve kombinált hővezetési és szilárdsági problémák megoldására is. Itt a gátolt hőtágulás okozta hőfeszültségek megállapítás az elsődleges cél. Mágneses és elektrosztatikus problémák szintén jól kezelhetők a módszerrel. Újabban gyakran használják a vízépítésben szivárgási és talajmechanikai vizsgálatok elvégzésére is.

A végeselemes módszer kézi számításokra alkalmatlan, mivel nagyon sok elemi műveletet kellene megoldani hozzá. A mai viszonylag kis teljesítményű személyi számítógépek is alkalmasak azonban, hogy egy sor gyakorlatilag fontos feladat megoldható legyen. A végeselemes módszer alkalmazásánál nemcsak a tulajdonképpeni matematikai feladat (egyenletrendszer megoldása, sajátérték-feladat) megoldása munkaigényes, hanem magának az adatoknak az előkészítése és az eredmények értékelhető alakba hozása is sok időt igényelne. Ezért a korszerű számítógépes programok részét képezi egy preprocesszor és egy posztprocesszor is. A gépalkatrész példa esetén a preprocesszor egy CAD programban elkészített rajzból vagy térbeli modellből kiindulva automatikusan vagy félautomatikusan generálja a végeselemes hálót és ugyancsak segít a terhelések felhasználóbarát megadásában is. A posztprocesszor pedig vagy felrajzolja a torzított léptékű deformált alakot (vagyis a deformációk értékét egy nagyobb számmal megszorozva jól láthatóvá teszi az alakváltozást) vagy szintvonalakkal vagy színezéssel bejelöli a különböző feszültségű tartományokat. Sok új CAD program már tartalmaz egyszerűbb végeselemes modult.

https://hu.wikipedia.org/wiki/V%C3%A9geselemes_m%C3%B3dszer



\Section{Részecske alapú megközelítés}

Gömbszerű részecskék

Egymással ütköznek

Tapadhatnak egymáshoz valamilyen szinten

\textbf{Részecske}

A részecske alapú megközelítés lényege, hogy a vizet, mint kis részecskék vagyis körök ábrázoljuk. A program működésének alapelve, hogy random ledobunk kis részecskéket és nézzük, hogy hol ütköznek, majd megvizsgáljuk, hogy az ütközés után merre kell haladniuk. 

Minden részecskének van színe, mérete, ami az összesre nézve azonos. 

A megvalósításhoz 3 osztályt használunk, ami a pont, részecske és maga a szimuláció. A pont osztályban a pontok tulajdonságai vannak. A részecske osztályban  a részecskék tulajdonságai és kirajzolása. A szimulátor osztályban pedig maga a szimulátor, a részecskék létrehozásával, a kép frissítéssel és az ütközésvizsgálattal. 

\Section{Cella alapú módszer}

Rácsfelbontás, most praktikusan egy négyzetrács

Mátrixban ábrázolható

Minden cellának van egy kitöltöttség értéke, például $[0, 1]$, ahol 0 az üres, 1 a teljesen kitöltött.

Jelölni kell a falakat is, például negatív vagy valamilyen nagyobb értékkel.
