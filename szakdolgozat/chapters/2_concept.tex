\Chapter{Koncepció}

\Section{Elterjedt szimulációs módszerek}

\Section{A szimuláció alapja}



Ha valamit szimulálunk az olyan, mintha utánoznánk \cite{szim}. Azonban a szimulációs módszerek mást jelentenek, mert azokkal egy rendszer viselkedését próbáljuk meghatározni. A modell alkalmazásával képesek vagyunk létrehozni állapot sorozatokat, amelyek leírják az adott rendszert vagy néhány komponensének viselkedését. 

A szimulációnak a használata ott javasolt, ahol

\begin{itemize}
	\item nincsennek analitikus módszerek vagy használatuk körülmények
	\item veszélyes és hosszadalmas lenne a valóságban tesztelés
	\item termeléskiesés miatt nem engedélyezett a valóságban tesztelni
	\item a kimenő eredmény nem észlelhető
\end{itemize}

Nagyon sok területen alkalmaznak szimulációt, logisztikai folyamatokon kívül, még használják pénzügyi, jármű és dinamikai modelleknél \cite{szim2}. 
Működés alapján 3 fajta modellt különböztetünk meg.
Ezek:

\begin{itemize}
	\item SD System Dynamics (rendszerdinamikai SD)
	\item  DES Discrete Event Simulation (Eseményvezérelt DES)
	\item  ABS (ágens alapú)
\end{itemize}

\Section{SD System Dynamics (rendszerdinamikai SD)}

Az összetevők változását nézi az idő függvényében. Ezt a módszert döntéshozási szinten használják logisztikai rendszerekben. 

Jellemzői:

\begin{itemize}
	\item minimális az adatszükséglet
	\item  struktúrára és stratégiára koncentrál
	\item  holisztikus megközelítés, alrendszerekkel integrálva
	\item biztosítja a vállalati stratégia rendszerbe épülését visszaható burok által
\end{itemize}

A hurok diagram a jellemző ábrázolási formája más néven CLD (Casual Loop Diagram).

A modell 3 fő elemből áll.
Ezek:

\begin{itemize}
	\item készletek
	\item  áramlások
	\item  késések
\end{itemize}


Ezek kölcsönösen kapcsolatban állnak, hatnak egymásra, illetve változnak.
A kimeneti érték a visszaható hurok sűrűségétől, illetve a hatás sebességétől függ. 
Gyakran alkalmazzák ezt a diagramot döntéshozáshoz, mert könnyen olvasható.

Alkotásakor először az egyes elemek között fennálló kapcsolatokat kell meghatározni. Majd ezekhez függvényeket, numerikus értékeket rendelni. Majd addig kell a futtatást pontosítani, amíg az érték nem tükrözi a valóságot. 



\Section{DES Discrete Event Simulation (Eseményvezérelt DES)}

Ez a modell állapotváltozók és események hálózata, ahol diszkrét időközönként az állapotváltozók állapotot változtatnak.
Az események bekövetkezésének időpontjai ezek a diszkrét időközök. Az események pedig pillanatnyi történések, amelyek megváltoztathatják a rendszer állapotát.
Ez az állapotváltozás valamilyen változást idéz elő a rendszer erőforrásaiban. Ennek a változásnak a mértéke egységekben mérhető.

\Section{ABS (ágens alapú)}

Ezek a szimulációs modellek legújabb csoportjába tartoznak. Azt az aktív szoftver modult nevezzük ágensnek, mely valamilyen beavatkozást végez a környezetéből beérkező információ hatására. 
Most tekintsük át a logisztikai alkalmazását. 

A strukturális változásoknak a modellezéséhez fejlesztették ki ezt a platformot. Az ágens alapú modell az ellátási láncban lévő virtuális vállalatok hatását szimulálja. Lényege, hogy egymástól eltérő ágenseket elhelyezünk, amelyek a virtuális vállalatokat reprezentálják. Ezeknek a vállalatoknak a tevékenységét szintén egy ágens szimulálja. Ezek különböző termékeket állítanak elő. Az összes ágens a saját termékei koncentrál, de közben egymásra is hatással vannak. Fontos, hogy ezek a logisztikai folyamatok a valóságban is ágensekből állnak, például csomagoló, targonca kezelő. 


\Section{Végeselem}



A véges elem módszerekkel
bonyolult feladatokat egyszerűsítünk le, ezt hálózásnak nevezzük \cite{veges1}. Ennek lényege, hogy az elemeket véges számú kisebb, egyszerűbb elemekre bontjuk, így több, de egyszerűbb számításokat kell elvégeznünk, a kevesebb, de bonyolultabb számítások helyett.


A módszer elterjedésével megváltoztak a gyártási folyamatok. Ezeket a módszereket jellemzően mérnökök használják, mechanikai számítások elvégzéséhez, valamint mezőszimulációkhoz \cite{wiki}. Tipikus példa egy bonyolult alakú statikusan terhelt gépalkatrész feszültségi állapotának és alakváltozásának meghatározása. Ekkor az alkatrész modellt véges számú elemre bontják. Ilyenek a síkbeliek közül a háromszögek és a négyszögek, térbeliek közül pedig a hasábok és tetraéderek. A felosztást úgy érdemes csinálni, hogy ahol az eredmény kritikus lehet ott kisebb részekre osztani, máshol pedig nagyobb részekre. Az elemek a csomópontoknál csatlakoznak egymáshoz. 


A véges elem módszert még nemlineáris feladatoknál is alkalmazni lehet. Például nemlineáris anyagtulajdonságok, nagy deformációk kezelése, szerkezeti stabilitási problémák. Hóvezetési problémák és kombinált hővezetési és szilárdsági problémák megoldására is használják a módszert. 


A mágneses és elektrosztatikus problémákra is megfelelő a használata. Mostanában szivárgási és talajmechanikai vizsgálatokat is végeznek vele. 

Azonban a véges elem módszer kézi számításokra nem alkalmas, mert nagyon sok elemi műveletet kellene végezni. A kis teljesítményű személyi számítógépek azonban alkalmasak, hogy megoldható legyen egy sor gyakorlatilag fontos feladat. Nem csak a matematikai feladat megoldása munkaigényes, hanem maga az adatok előkészítése, és az eredmények értékelhető alakba hozása is. Ezért egy preprocesszor és egy posztprocesszor is a korszerű számítógépes programok részét képezik.   




\Section{Részecske alapú megközelítés}


A részecske alapú megközelítés lényege röviden, hogy a vizet, mint kis részecskék vagyis körök/négyzetek ábrázoljuk. A program működésének alapelve, hogy random ledobunk/megjelenítünk kis részecskéket, és nézzük, hogy hol ütköznek, majd megvizsgáljuk, hogy az ütközés után merre kell haladniuk. 

Minden részecskének van x, y koordinátája, sugara/magassága, szélessége, színe, sebesség vektora, visszapattanási együtthatója és mérete. 

A megvalósításhoz mindkét esetben 3 osztályt használunk. Ami a kör alapúban a pont, részecske és maga a szimuláció. A pont osztályban a pontok tulajdonságai vannak, koordináták, eltolás és távolságszámítás.  A részecske osztályban  a részecskék tulajdonságai (koordináták, sugár, szín, mozgás vektor, méret és visszapattanási együttható) és egy részecske kirajzolása. A szimulátor osztályban pedig maga a szimulátor, a részecskék létrehozásával, a kép frissítéssel és az ütközésvizsgálatokkal, ami külön van a fallal való ütközésre és a részecskék közöttire, illetve a programban van úgynevezett pohár, amihez szintén szükséges ütközésvizsgálat. A pohár kirajzolása is ebben a részben történik.

A négyzet alapúban pedig, a rács, a pixel és a szimuláció. A rács osztályban a sorok, oszlopok száma van. A pixel osztályban a pixelek tulajdonságai (koordináták, szín, szélesség, magasság, méret, mozgás vektor és visszapattanási együttható), illetve távolságszámítás, és egy részecske kirajzolása. A szimulátor osztályban, pedig itt is ugyanazok találhatóak, mint a kör alapúban. 

