\Chapter{Összefoglalás}



\Section{Kezdetek}

Összegezve a dolgozat először bemutatta egy kicsit a webböngészőket és valamennyire a bennük rejlő lehetőségek is említésre kerültek. Ez azért volt szükséges mivel a program böngészőben fut. 

Majd áttértünk a szimulációra, ahol először magáról a szimuláció jelentéséről esett szó. Utána pedig a különböző szimulációs módszerekről/fajtákról. Említésre került még a végeselem módszer is. Mivel az is egy úgymond szimulációs megoldás. 

\Section{Program}

Aztán jött magának a programnak a bemutatása. A kirajzolásokkal, ütközésvizsgálatokkal. 

Ezek közül a kirajzolások, és az ütközések egy része nem úgy sikerült, ahogy terveztem. 

De vannak olyan részek is, amelyek jobban sikerültek a tervezettnél. 


Összegezve a program próbálja bemutatni a valós idejű vízmodellezés egyik fajtáját. 

A részecskéket próbálja valós idejűen mozgatni, a különböző földi tényezőkkel számolva. 

Az ütközések is próbálnak a valós ütközésekhez hasonlítani. 

\Section{Jövőbeni fejlesztések}

Ezt a programot/témát valószínűleg még elég rendesen lehetne fejleszteni, módosítani. 

Lehetne belerakni többfajta akadályt. Több tényezőt, ami hathat a részecskékre, ilyen lehet akár a szél. 

A futási időn is lehetne módosítani különböző vizsgálatokkal. 

Amiket ebből a programból még ki lehetne hozni, azokat majd idővel szeretném is megvalósítani. 
