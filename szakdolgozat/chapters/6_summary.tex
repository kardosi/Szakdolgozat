\Chapter{Összefoglalás}



Összegezve a dolgozat először bemutatta egy kicsit a webböngészőket és valamennyire a bennük rejlő lehetőségek is említésre kerültek. Ez azért volt szükséges mivel a program böngészőben fut. Illetve itt már említésre került a programban használt modell is. 

Majd áttértünk a szimulációra, ahol először magáról a szimuláció jelentéséről esett szó. Utána pedig a különböző szimulációs módszerekről/fajtákról. Említésre került még a véges elem módszer. Mivel az is egy úgymond szimulációs megoldás. Ebben a részben már kicsit részletesebben is ki lett fejtve az általunk használt modell.

Aztán jön a programnak a tervezése. Itt leírásra kerültek azok a dolgok, amik szerepelnek a programban. 


Majd ezek után jött a kész programnak a bemutatása. A kirajzolásokkal, ütközésvizsgálatokkal. 

Szó esett még a program írása közben lezajló tesztelésekről is, amikből nem volt sok, de viszont fontosak a program megírásában. 


Most lássuk, hogy szerintem melyik részek, hogyan sikerültek. Ezek közül a kirajzolások, és az ütközések egy része nem úgy sikerült, ahogy terveztem. 

De vannak olyan részek is, amelyek jobban sikerültek a tervezettnél. 


Összegezve a program próbálja bemutatni a valós idejű vízmodellezés egyik fajtáját. 

A részecskéket próbálja valós idejűen mozgatni, a különböző földi tényezőkkel számolva. 

Az ütközések is próbálnak a valós ütközésekhez hasonlítani. 

A jövőbe nézve ezt a programot/témát valószínűleg még lehetséges lesz fejleszteni/módosítani.

Lehetne belerakni még többfajta akadályt. Több tényezőt, ami hathat a részecskékre, ilyen lehet akár a szél. 

A futási időn is lehetne módosítani különböző vizsgálatokkal. 

Amiket ebből a programból még ki lehetne hozni, azokat majd idővel szeretném is megvalósítani. 
