\pagestyle{empty}

\noindent \textbf{\Large CD Használati útmutató}

\vskip 1cm

%\textit{A mellékelt CD tartalma} 

A CD-n két mappa található egy szakdolgozat és egy simulators. A szakdolgozat mappában van magának a dolgozatnak a szöveges része. Ezen belül van még 4 mappa a chapters, cover, images, és styles. A chapters mappában vannak a dolgozat fejezeteihez tartozó tex fájlok.. A cover-ben pedig a címlaphoz, feladatkiíráshoz, illetve a használati útmutatóhoz tartozó tex fájlok. Az images mappában, pedig a dolgozatban használt képek találhatóak meg. A styles mappában pedig a dolgozatnál használt stílusok. Ezeken kívül vannak még különböző fájlok is a szakdolgozat mappában. Itt található meg a dolgozathoz tartozó pdf fájl is. Illetve az a fájl, amely az irodalomjegyzéket tartalmazza. És még a dolgozathoz tartozó tex fájl is itt van. 

\bigskip

A simulátorsban 2 mappa található ezek a cells és a particles. A cells mappában a négyzet alapú program részeit találjuk. A javascript fájlokat, a css-t illetve a HTML-t. 
A particles osztályban a kör alapú program részeit találjuk. Szintén a hozzá tartozó javascript, css és HTML fájlok. 
Amik még a simulators mappában vannak azok pedig draw és a draw2 a jquery tesztekhez készült HTML fájlok és itt van még azoknak az eredményeit illetve diagramjait tartalmazó excel fájlok. Van itt még egy Diagram1.dia és Diagram2.dia fájl ezek pedig a programokhoz tartozó UML diagramokat tartalmazzák. Illetve a jqery futtatásához szükséges fájlok (jquery.js illetve jcanvas.min.js).